%% %%%%%%%%%%%%%%%%%%%%%%%%%%%%%%%%%%%%%%%%%%%%%%%%%%%%%%%%%%%%%%%%%%%%%%%%%%%%%
%% ltuthesis.cls Example
%% =============================================================================
%% This is an example LaTeX file that shows how the 'ltuthesis.cls' file is 
%% used. Most of the code should be self-explanatory. There are a couple of
%% known issues in the template, have a look at the top of the class file and/or
%% the TODO.md file.
%% 
%% 2012-05-10 / Roland Hostettler <rolhos@ltu.se>
%% %%%%%%%%%%%%%%%%%%%%%%%%%%%%%%%%%%%%%%%%%%%%%%%%%%%%%%%%%%%%%%%%%%%%%%%%%%%%%
%\documentclass[twoside,openright,12pt,a4paper]{ltuthesis}
\documentclass[twoside,openright,12pt,a4paper,thumbmarks]{ltuthesis}

%% %%%%%%%%%%%%%%%%%%%%%%%%%%%%%%%%%%%%%%%%%%%%%%%%%%%%%%%%%%%%%%%%%%%%%%%%%%%%%
%% Preamble - Include and Setup Packages
%% %%%%%%%%%%%%%%%%%%%%%%%%%%%%%%%%%%%%%%%%%%%%%%%%%%%%%%%%%%%%%%%%%%%%%%%%%%%%%
%% Input encoding: Most modern OSes use UTF-8 and the template is written in
%% UTF-8, however, others should work too
\usepackage[utf8]{inputenc}

%% Hyperref: Practical if you want a PDF with links and an index (but not
%% required). Note that the library will unfortunately remove that when they
%% add the cover pages
\usepackage[plainpages=false,pdfpagelabels]{hyperref}
\hypersetup{
%    bookmarks=true,         		% show bookmarks bar?
    unicode=false,          		% non-Latin characters bookmarks
    pdftoolbar=true,        		% show Acrobat's toolbar?
    pdfmenubar=true,        		% show Acrobat's menu?
    pdffitwindow=false,     		% window fit to page when opened
    pdfstartview={FitH},    		% fits the width of the page to the window
    pdftitle={Vehicle Parameter Estimation using Road Surface Vibrations},					    % title
    pdfauthor={Roland Hostettler},	% author
    pdfsubject={},   				% subject of the document
    pdfcreator={pdfTeX},			% creator of the document
    pdfproducer={}, 				% producer of the document
    pdfkeywords={},					% list of keywords
    pdfnewwindow=true,  			% links in new window
    colorlinks=true,				% false: boxed links; true: colored links
    linkcolor=black,    			% color of internal links
    citecolor=black,    			% color of links to bibliography
    filecolor=black,    			% color of file links
    urlcolor=black,      			% color of external links
}

%% If you have appendices in one (or more) of your articles, you need the
%% following lines
%% TODO: See how we can properly integrate this in the template
\usepackage{appendix}
\renewcommand{\setthesection}{\Alph{section}}
\renewcommand*{\theHchapter}{\thepart.\thechapter}

\usepackage{lipsum}


%% %%%%%%%%%%%%%%%%%%%%%%%%%%%%%%%%%%%%%%%%%%%%%%%%%%%%%%%%%%%%%%%%%%%%%%%%%%%%%
%% Start the document
%% %%%%%%%%%%%%%%%%%%%%%%%%%%%%%%%%%%%%%%%%%%%%%%%%%%%%%%%%%%%%%%%%%%%%%%%%%%%%%
\begin{document}

%% %%%%%%%%%%%%%%%%%%%%%%%%%%%%%%%%%%%%%%%%%%%%%%%%%%%%%%%%%%%%%%%%%%%%%%%%%%%%%
%% Setup the Bibliography
%% %%%%%%%%%%%%%%%%%%%%%%%%%%%%%%%%%%%%%%%%%%%%%%%%%%%%%%%%%%%%%%%%%%%%%%%%%%%%%
%% The bibliography is handled by bibunits which creates individual 
%% bibliography units for the "Kappa"-part and one for each of the papers. The
%% bibunits are created automatically and the bibliography for the kappa is 
%% added automatically too. However, in the papers you have to issue a '\putbib'
%% command at the place where you want the bibliography to appear.
%%
%% '\defaultbibliograph' and '\defaultbibliographystyle' define the default
%% bibliography file and style used for the entire document if no other
%% bibliography and/or style is specified within the respective bibunit.
%%
%% Finally, bibunits creates a set of bibliography files called 'buX.aux', one 
%% for each bibunit. You have to run bibtex on each of them to genereate the
%% corresponding bbl-file. On Linux I use the following script (which is
%% involved as a post-processor after pdflatex in my editor:
%%
%%     # !/bin/bash
%%    for file in bu*.aux; do
%%        bibtex $file
%%    done;
%%
%% Details about all this can be found in the bibunits manual.
\defaultbibliography{dummy}
\defaultbibliographystyle{IEEEtran}


%% %%%%%%%%%%%%%%%%%%%%%%%%%%%%%%%%%%%%%%%%%%%%%%%%%%%%%%%%%%%%%%%%%%%%%%%%%%%%%
%% Front Matter
%% %%%%%%%%%%%%%%%%%%%%%%%%%%%%%%%%%%%%%%%%%%%%%%%%%%%%%%%%%%%%%%%%%%%%%%%%%%%%%
%% This part should be pretty self-explanatory...
\title{Alternative CSEE Thesis Template}
\author{Roland Hostettler}
\address{Department of Computer Science, Electrical and Space Engineering\\
	Luleå University of Technology\\
	Luleå, Sweden}
\supervisors{Your Supervisor}
\maketitle

%% Dedication, optional
\begin{dedication}
	Your dedication
\end{dedication}

%% Abstract
\begin{abstract}
Your thesis abstract.
\end{abstract}

%% Table of Contents
\tableofcontents

%% Acknowledgments, optional
\begin{acknowledgments}
Your thesis acknowledgments.
\end{acknowledgments}


%% %%%%%%%%%%%%%%%%%%%%%%%%%%%%%%%%%%%%%%%%%%%%%%%%%%%%%%%%%%%%%%%%%%%%%%%%%%%%%
%% Kappa
%% %%%%%%%%%%%%%%%%%%%%%%%%%%%%%%%%%%%%%%%%%%%%%%%%%%%%%%%%%%%%%%%%%%%%%%%%%%%%%
%% The 'kappa' part of the thesis is put into the 'intro' environment. This
%% environment makes sure that all the formatting, etc. is setup correctly.
%% After that it's just a matter of including the different chapters.
\begin{intro}
\chapter{Introduction}
\label{chap:introduction}
This is the manual for the \texttt{ltuthesis} LaTeX template. The template's goal is to provide a headache-free way of integrating the thesis introduction (``Kappa'') and the research papers in one nice-looking, consistent thesis. Even though the template arose from writing composite theses, it can equally well be used for monographs -- see Chapter~\ref{chap:tips_tricks}.

This manual documents the usage of the thesis template with its different bells and whistles providing a complete documentation and guide on how to write your thesis by using this manual. Furthermore, the manual itself is written in the template and can be used as a starting point of your thesis. The source file of this document is filled with lots of comments that explain how the different commands are used.

The template tries to adhere to standard \LaTeX practice and only use techniques that are considered ``the right way'' of doing things in \LaTeX. However, {\color{red} say something blubberish.}

The orgianization of this manual is as follows. Chapter~\ref{chap:usage} highlights the most basic usage of the template and should be read by everyone using this template. Chapter~\ref{chap:tips_tricks} includes some tips and tricks and advanced features of the template, workspace organization, and further topics. In Part II, {\color{red} XX} example research papers are included. Each of the research papers is an example of a paper in the different stages of the academic publishing process.

Finally, {\color{red} a note about warranty and license}.

\chapter{Usage}
\label{chap:usage}
A composite thesis at LTU normally consists of three parts:
%
\begin{enumerate}
    \item preface;
    \item thesis introduction (``Kappa'');
    \item research papers.
\end{enumerate}
%
A description of how these three parts are integrated in the thesis template follows in their individual sections below. However, before we can get started, we need to define our thesis to be of the \texttt{ltuthesis} document class:
%
\begin{lstlisting}
    \documentclass[twoside,openright,12pt,a4paper]{ltuthesis}
\end{lstlisting}
%
This will load the template with a font size 12 points, A4 paper\footnote{Your thesis will be printed in the S5 format, not A4. However, the printing office guidelines state that ``{\color{red} add quote here}'', see~\cite{blabla}}, double-sided layout, and new chapters starting on a right page. There are no additional packages required for you to be able to use the template (you might, however, want to have a look into Chapter~\ref{chap:tips_tricks} for a couple of common packages that one might want to use).


%% %%%%%%%%%%%%%%%%%%%%%%%%%%%%%%%%%%%%%%%%%%%%%%%%%%%%%%%%%%%%%%%%%%%%%%%%%%%%%
\section{Preface}
\label{sec:usage:preface}
The preface of your thesis includes the following parts:
%
\begin{enumerate}
    \item Title page;
    \item dedication or quote (optional);
    \item abstract;
    \item table of contents;
    \item acknowledgments (optional).
\end{enumerate}
%
Note that by the term ``title page'' we refer to the title page inside the printed book, and not the cover page (which will be added by the university library).

\paragraph{Title Page} The title page includes the general information about your thesis, such as thesis title, your name, organization, and supervisors. It is created by setting the corresponding values in the preamble of your document as follows:
%
\begin{lstlisting}
\title{Your Thesis Title}
\author{Your Name}
\address{Your Affiliation}
\supervisors{Your Supervisor(s)}
\end{lstlisting}
%
This is followed by a simple call to \texttt{maketitle} in your document body
%
\begin{lstlisting}
\maketitle
\end{lstlisting}
%
which will put all the information in place.

\paragraph{Dedication} Next, it is common to dedicate your thesis to someone or to include a quote or something similar. You simply do this by including it in the \texttt{dedication}-environment as follows:
%
\begin{lstlisting}
\begin{dedication}
    Your dedication here.
\end{dedication}
\end{lstlisting}
%
In general, a dedication is not necessary but customary.

\paragraph{Abstract} Similar to the dedication, the thesis abstract is placed in an \texttt{abstract}-environment:
%
\begin{lstlisting}
\begin{abstract}
    Your abstract here.
\end{abstract}
\end{lstlisting}

\paragraph{Table of Contents} Next up is the table of contents. As in other latex classes, you simply have to make a call to
%
\begin{lstlisting}
\tableofcontents
\end{lstlisting}
%
to include it in your thesis.

\paragraph{Acknowledgments} Finally, you might want to acknowledge a couple of people that helped you during your Ph.D.\ studies and life in general. Similar to the previous cases (you might already suspect it), put your acknowledgments into an \texttt{acknowledgments}-environment:
%
\begin{lstlisting}
\begin{acknowledgments}
    Put your acknowledgments here.
\end{acknowledgments}
\end{lstlisting}

This concludes the preface of your thesis and everything should be in place. You should now be able to compile the bare minimum of your thesis by using \texttt{pdflatex}:
%
\begin{verbatim}
pdflatex thesis.tex
\end{verbatim}


%% %%%%%%%%%%%%%%%%%%%%%%%%%%%%%%%%%%%%%%%%%%%%%%%%%%%%%%%%%%%%%%%%%%%%%%%%%%%%%
\section{Kappa}



%% %%%%%%%%%%%%%%%%%%%%%%%%%%%%%%%%%%%%%%%%%%%%%%%%%%%%%%%%%%%%%%%%%%%%%%%%%%%%%
\section{Papers}

include

labels

paths

bibliography


%% %%%%%%%%%%%%%%%%%%%%%%%%%%%%%%%%%%%%%%%%%%%%%%%%%%%%%%%%%%%%%%%%%%%%%%%%%%%%%
\section{Bibliography}
The bibliography of your thesis deserves some special attention and it is very important that you \textbf{read this section carefully}.



\end{intro}


%% %%%%%%%%%%%%%%%%%%%%%%%%%%%%%%%%%%%%%%%%%%%%%%%%%%%%%%%%%%%%%%%%%%%%%%%%%%%%%
%% Papers
%% %%%%%%%%%%%%%%%%%%%%%%%%%%%%%%%%%%%%%%%%%%%%%%%%%%%%%%%%%%%%%%%%%%%%%%%%%%%%%
%% Similarly to the 'intro'-environment, the 'papers'-environment sets up the
%% whole paper part with all the necessary formatting, etc.
\begin{papers}

%% Papers are included using the '\includepaper'-command. It uses the key-value
%% options format similar to many other LaTeX commands (e.g. \includegraphics).
%% The keys are:
%% 
%%     * title        The paper title
%%     * author       The paper author(s), use curly brackets for a comma-
%%                    separated list of authors
%%     * journal      The journal where the article was published
%%     * conference   The conference where the article was published
%%     * copyright    The copyright holder
%%     * published    For a published paper
%%     * accepted     For an accepted paper
%%     * submitted    For a submitted paper
%%
%% The main argument to the \includepaper-command is then the paper's main
%% file. This is supposed to be very slim, see Example/PaperA/main.tex or
%% Example/PaperB/main.tex. All the titles, etc. are typeset automatically.

%% Paper A: First Example Paper
\includepaper[title=An Example Paper,
	author={Author A and Author B},
	journal=A Journal of Important Studies,
	copyright={2012, The Institution of Important Studies},
	published]{PaperA/main}

%% Paper B: Second Example Paper
\includepaper[title=Another Example Paper,
	author={Author A, Author B, and Author C},
	conference={1st Conference on \LaTeX~Templates},
	copyright={2012, International \LaTeX~Federation},
	published]{PaperB/main}

\end{papers}
\end{document}


%% %%%%%%%%%%%%%%%%%%%%%%%%%%%%%%%%%%%%%%%%%%%%%%%%%%%%%%%%%%%%%%%%%%%%%%%%%%%%%
%% EOF
%% %%%%%%%%%%%%%%%%%%%%%%%%%%%%%%%%%%%%%%%%%%%%%%%%%%%%%%%%%%%%%%%%%%%%%%%%%%%%%

