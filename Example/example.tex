% ltuthesis.cls Example %%%%%%%%%%%%%%%%%%%%%%%%%%%%%%%%%%%%%%%%%%%%%%%%%%%%%%%%
% This is an example and documentation to show and document the 'ltuthesis.cls'
% LaTeX class. It can be used as a starting point for writing your own thesis.
%
% IMPORTANT NOTICE: The template and its features are thoroughly described in
% the PDF produced by this example. Please refer to this documentation first.
%
% In order to build the example, run ...
%
% TODO 
%   [ ] Include the commands required to compile this example. Also, include a
%       makefile.
%   [ ] I might want to include the option of creating the library-style front-
%       page in order to publish it myself.
%
% 2014-06-04 -- Roland Hostettler <roland.hostettler@ltu.se>

% Document Class Definition %%%%%%%%%%%%%%%%%%%%%%%%%%%%%%%%%%%%%%%%%%%%%%%%%%%%
% Define the document class to use. Just as for the book class, we define a two-
% sided layout where chapters start on right-handed pages. Furthermore, we use 
% A4 paper and 12pt fonts. Of course other options can be used.
%
% Finally, the 'thumbmarks' option instructs the template to place thumbmarks on
% for easy navigation (optional of course).
%\documentclass[twoside,openright,12pt,a4paper]{ltuthesis}
\documentclass[twoside,openright,12pt,a4paper,thumbmarks]{ltuthesis}


% Load Packages %%%%%%%%%%%%%%%%%%%%%%%%%%%%%%%%%%%%%%%%%%%%%%%%%%%%%%%%%%%%%%%%
% Here, we load the packages required by the thesis. None of the packages is re-
% quired by the template as the required packages are loaded by the class file 
% itself. However, the packages loaded below give you a good starting point.

% Input encoding
% Most modern OSes use UTF-8 and the template is written in UTF-8, however, 
% others should work too.
\usepackage[utf8]{inputenc}

% To fix block alignment
\usepackage{microtype}

% hyperref
% Practical if you want a PDF with links and an index.
\usepackage[plainpages=false,pdfpagelabels]{hyperref}
\hypersetup{
    % General appearance when opening the PDF
    % bookmarks=true,% show bookmarks bar?
    unicode=false,                                % Use non-Latin characters in bookmarks?
    pdftoolbar=true,                              % Show Acrobat's toolbar?
    pdfmenubar=true,                              % show Acrobat's menu?
    pdffitwindow=false,                           % Window fit to page when opened
    pdfstartview={FitH},                          % Fits the width of the page to the window
    %    
    % Meta information
    pdftitle={Alternative LTU Ph.D. Thesis Template}, % PDF Title
    pdfauthor={Roland Hostettler},                % Author
    pdfsubject={},                                % Subject of the document
    pdfcreator={pdfTeX},                          % Creator of the document
    pdfproducer={},                               % Producer of the document
    pdfkeywords={},                               % List of keywords
    %
    % Appearance and behavior of links
    pdfnewwindow=true,                            % Links in new window
    colorlinks=true,                              % False: boxed links; true: colored links
    linkcolor=black,                              % Color of internal links
    citecolor=black,                              % Color of links to bibliography
    filecolor=black,                              % Color of file links
    urlcolor=black,                               % Color of external links
}


% appendix
% If you have appendices in your articles, you can use the 'appendix' package
% to make them appear in the correct format. Include the following lines:
\usepackage{appendix}
\renewcommand{\setthesection}{\Alph{section}}
\renewcommand*{\theHchapter}{\thepart.\thechapter}


% lipsum
% This package is to generate (realistic) arbitrary text, simply remove it.
\usepackage{lipsum}
\usepackage{listings}
\lstset{language=[LaTeX]TeX}

\usepackage{color}


% Document Start %%%%%%%%%%%%%%%%%%%%%%%%%%%%%%%%%%%%%%%%%%%%%%%%%%%%%%%%%%%%%%%
\begin{document}

% Bibliography Setup %%%%%%%%%%%%%%%%%%%%%%%%%%%%%%%%%%%%%%%%%%%%%%%%%%%%%%%%%%%
% The bibliography is handled by the 'bibunits' package which creates individual
% bibliographies, one for the kappa and one for each of the papers. These are 
% created automagically and the bibliography for the kappa is added 
% automatically after the last chapter of the kappa. However, in the papers you
% have to issue a '\putbib' command at the place where you want the bibliography
% to appear, see the example papers included.
%
% '\defaultbibliograph' and '\defaultbibliographystyle' define the default
% bibliography file and style used for the entire document if no other
% bibliography and/or style is specified within the respective bibunit. The 
% latter is advanced and the user is referred to the bibunits documentation.
%
% Finally, bibunits creates a set of bibliography files called 'buX.aux', one 
% for each bibunit. You have to run bibtex on each of them to genereate the
% corresponding bbl-file. On Linux I use the following script (which is
% involved as a post-processor after pdflatex in my editor:
%
%     # !/bin/bash
%    for file in bu*.aux; do
%        bibtex $file
%    done;
%
% Details about all this can be found in the bibunits manual.
\defaultbibliography{example}
\defaultbibliographystyle{IEEEtran}


% Preface %%%%%%%%%%%%%%%%%%%%%%%%%%%%%%%%%%%%%%%%%%%%%%%%%%%%%%%%%%%%%%%%%%%%%%
% The following commands setup the preface of the thesis which consists of:
%
%   1. Title page
%   2. Dedication (optional)
%   3. Abstract
%   4. Table of Contents
%   5. Acknowledgments

% Title and affilation
% This part should be pretty self-explanatory...
\title{Alternative LTU Ph.D.\ Thesis Template}
\author{Roland Hostettler}
\address{Department of Computer Science, Electrical and Space Engineering\\
	Division of Signals and Systems\\
	Luleå University of Technology\\
	Luleå, Sweden}
\supervisors{Your Supervisor}
\maketitle

% Dedication, optional
\begin{dedication}
    Your dedication
\end{dedication}

%% Abstract
\begin{abstract}
    \input{Preface/abstract}
\end{abstract}

%% Table of Contents
\tableofcontents

%% Acknowledgments, optional
\begin{acknowledgments}
    \input{Preface/acknowledgments}
\end{acknowledgments}


% Kappa %%%%%%%%%%%%%%%%%%%%%%%%%%%%%%%%%%%%%%%%%%%%%%%%%%%%%%%%%%%%%%%%%%%%%%%%
% The kappa of the theisis is put into the 'intro' environment as shown below.
% This ensures that all the formatting, etc. is setup correctly and it's just a
% matter of including the different chapters. 
\begin{intro}
    \chapter{Introduction}
\label{chap:introduction}
\lipsum

    \chapter{Usage}
\label{chap:usage}
A composite thesis at LTU normally consists of three parts:
%
\begin{enumerate}
    \item preface;
    \item thesis introduction (``Kappa'');
    \item research papers.
\end{enumerate}
%
A description of how these three parts are integrated in the thesis template follows in their individual sections below. However, before we can get started, we need to define our thesis to be of the \texttt{ltuthesis} document class:
%
\begin{lstlisting}
    \documentclass[twoside,openright,12pt,a4paper]{ltuthesis}
\end{lstlisting}
%
This will load the template with a font size 12 points, A4 paper\footnote{Your thesis will be printed in the S5 format, not A4. However, the printing office guidelines state that ``{\color{red} add quote here}'', see~\cite{blabla}}, double-sided layout, and new chapters starting on a right page. There are no additional packages required for you to be able to use the template (you might, however, want to have a look into Chapter~\ref{chap:tips_tricks} for a couple of common packages that one might want to use).


%% %%%%%%%%%%%%%%%%%%%%%%%%%%%%%%%%%%%%%%%%%%%%%%%%%%%%%%%%%%%%%%%%%%%%%%%%%%%%%
\section{Preface}
\label{sec:usage:preface}
The preface of your thesis includes the following parts:
%
\begin{enumerate}
    \item Title page;
    \item dedication or quote (optional);
    \item abstract;
    \item table of contents;
    \item acknowledgments (optional).
\end{enumerate}
%
Note that by the term ``title page'' we refer to the title page inside the printed book, and not the cover page (which will be added by the university library).

\paragraph{Title Page} The title page includes the general information about your thesis, such as thesis title, your name, organization, and supervisors. It is created by setting the corresponding values in the preamble of your document as follows:
%
\begin{lstlisting}
\title{Your Thesis Title}
\author{Your Name}
\address{Your Affiliation}
\supervisors{Your Supervisor(s)}
\end{lstlisting}
%
This is followed by a simple call to \texttt{maketitle} in your document body
%
\begin{lstlisting}
\maketitle
\end{lstlisting}
%
which will put all the information in place.

\paragraph{Dedication} Next, it is common to dedicate your thesis to someone or to include a quote or something similar. You simply do this by including it in the \texttt{dedication}-environment as follows:
%
\begin{lstlisting}
\begin{dedication}
    Your dedication here.
\end{dedication}
\end{lstlisting}
%
In general, a dedication is not necessary but customary.

\paragraph{Abstract} Similar to the dedication, the thesis abstract is placed in an \texttt{abstract}-environment:
%
\begin{lstlisting}
\begin{abstract}
    Your abstract here.
\end{abstract}
\end{lstlisting}

\paragraph{Table of Contents} Next up is the table of contents. As in other latex classes, you simply have to make a call to
%
\begin{lstlisting}
\tableofcontents
\end{lstlisting}
%
to include it in your thesis.

\paragraph{Acknowledgments} Finally, you might want to acknowledge a couple of people that helped you during your Ph.D.\ studies and life in general. Similar to the previous cases (you might already suspect it), put your acknowledgments into an \texttt{acknowledgments}-environment:
%
\begin{lstlisting}
\begin{acknowledgments}
    Put your acknowledgments here.
\end{acknowledgments}
\end{lstlisting}

This concludes the preface of your thesis and everything should be in place. You should now be able to compile the bare minimum of your thesis by using \texttt{pdflatex}:
%
\begin{verbatim}
pdflatex thesis.tex
\end{verbatim}


%% %%%%%%%%%%%%%%%%%%%%%%%%%%%%%%%%%%%%%%%%%%%%%%%%%%%%%%%%%%%%%%%%%%%%%%%%%%%%%
\section{Kappa}



%% %%%%%%%%%%%%%%%%%%%%%%%%%%%%%%%%%%%%%%%%%%%%%%%%%%%%%%%%%%%%%%%%%%%%%%%%%%%%%
\section{Papers}

include

labels

paths

bibliography


%% %%%%%%%%%%%%%%%%%%%%%%%%%%%%%%%%%%%%%%%%%%%%%%%%%%%%%%%%%%%%%%%%%%%%%%%%%%%%%
\section{Bibliography}
The bibliography of your thesis deserves some special attention and it is very important that you \textbf{read this section carefully}.



    \chapter{Tips \& Tricks}
\label{chap:tips_tricks}

hyperref

character encoding

thumbmarks

subappendices

monographs

workspace

subfig vs. subcaption

other stuff like notation and so?

\end{intro}


% Papers %%%%%%%%%%%%%%%%%%%%%%%%%%%%%%%%%%%%%%%%%%%%%%%%%%%%%%%%%%%%%%%%%%%%%%%
% Similarly to the 'intro'-environment, the included papers are put into a 
% 'papers'-environment. Again, this sets up all the formatting, counters, etc.
\begin{papers}

% Papers themselves are included using the '\includepaper'-command. It uses the 
% key-value options format as used in many other LaTeX commands (for example 
% \includegraphics). Some of the keys are required while others complement each
% other. Different examples are given below.
%
% The keys are:
%   * title        The paper title
%   * author       The paper author(s), use curly brackets for a comma-
%                  separated list of authors
%   * journal      The journal where the article was/is going to be published
%                  (only if the paper is a journal paper)
%   * conference   The conference where the article was/is going to be published
%                  (only if the paper is a conference paper)
%   * copyright    The copyright holder (only if the copyright has been trans-
%                  ferred).
%   * published    For a published paper.
%   * accepted     For an accepted paper.
%   * submitted    For a submitted paper.
%   * doi          The digital object identifier (DOI), for published papers
%                  papers only.
%
% The main argument to the \includepaper-command is then the paper's main
% file. This is supposed to be very slim, see Example/Paper A/main.tex or
% Example/Paper B/main.tex. All the titles, etc. are typeset automatically.

% Example Paper A: Published conference paper
\includepaper[title={The First Example Paper},
    author={Author A and Author B},
    conference={54th Conference on LaTeX Templates},
    copyright={2012, The Institute of All Things LaTeX},
    published,
    doi={2012/ABC}]{PaperA/main}

% Example Paper B: Accepted journal paper
\includepaper[title={The Second Example Paper},
	author={Author A, Author C, and Author B},
	journal={The International Journal of LaTeX Templates},
	copyright={2012, The Institute of All Things LaTeX},
	accepted]{PaperB/main}

% Example Paper C: Submitted conference paper
\includepaper[title={The Third Example Paper},
	author={Author A, Author B, and Author C},
	conference={1st Symposium on LaTeX Templates},
	submitted]{PaperC/main}

\end{papers}
\end{document}

